
%%% Local Variables:
%%% mode: latex
%%% TeX-master: "../cryptography"
%%% End:

\chapter{Discrete Logarithms}

\section{Baby-step giant-step}

\begin{algorithm}[H]
  \caption{Shanks's Babystep-Giantstep Algorithm}
  \begin{algorithmic}
   \Require $g^x = b$
   \State $n \gets 1 + \floor{\sqrt{N}}$ 
   \State $b \gets \{e, g, g^2, \dots, g^n\}$
   \For $j \in \{0,1,\dots,n\}$
     \State $\text{table}[g^j] = j$
   \EndFor
   \State $\gamma \gets b$
   \ForAll{$i \in \{0,1,\dots,n\}$}
     \If{$\gamma \in \text{table}$}
       \State \textbf{return} $in + \text{table}[\gamma]$
     \Else
       \State $\gamma \gets \gamma \cdot g^{-n}$
     \EndIf
   \EndFor
     
  \end{algorithmic}
\end{algorithm}

\subsubsection{Correctness}

To prove that we always find a solution we need to show that there is always a match.

\begin{proof}
  We rewrite $x$ as follows:
  \[ x = nq + r, 0 \leq r < n \]
  We can solve for q and since $n > \sqrt{N}$ we obtain:
  \[ q = \frac{x-r}{n} < \frac{N}{n} < \frac{N}{\sqrt{N}} = \sqrt{N} < n \]
  Rewriting the discrete logarithm we get:
  \[ g^r = b\cdot g^{-nq} \]
  And since $0 \leq r < n$, $g^r$ is in the babysteps and because $q < n$, $-nq < -n^2$, so $b\cdot g^{-nq}$ is in the giant steps, therefore we always find a solution.
  
\end{proof}

\subsubsection{Time/Space Complexity}
Since we have $\mathcal{O}(1)$ lookup into the hashtable we have time complexity $\mathcal{O}(n) = \mathcal{O}(\sqrt{n})$. This consists of $2n$ modular exponentiations and $n$ hash table inserts and lookups respectively.

Regarding space, we have a hash table storing $n + 1$ elements, therefore we have $\mathcal{O}(\sqrt{n})$ space complexity.

\section{Pohlig-Hellman Algorithm}
\section{Index Calculus Algorithm}