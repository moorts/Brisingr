
%%% Local Variables:
%%% mode: latex
%%% TeX-master: "../cryptography"
%%% End:

\chapter{Discrete Logarithms}

A discrete logarithm of an element $b$ in a finite, cyclic group $G$, to a base $g \in G$ is a solution to:

\[ g^x \equiv b. \]

There are many algorithms to calculate discrete logarithms, however if the order of $g$ is large enough none of them suffice.

\section{Baby-step giant-step}

This algorithm is good if the order of $g$ is a prime, otherwise the pohlig-hellman algorithm is more efficient.

\begin{algorithm}[H]
  \caption{Shanks's Babystep-Giantstep Algorithm}
  \begin{algorithmic}
   \Require $g^x = h$
   \State $n \gets 1 + \floor{\sqrt{N}}$ 
   \State $h \gets \{e, g, g^2, \dots, g^n\}$
   \For $j \in \{0,1,\dots,n\}$
     \State $\text{table}[g^j] = j$
   \EndFor
   \State $\gamma \gets h$
   \ForAll{$i \in \{0,1,\dots,n\}$}
     \If{$\gamma \in \text{table}$}
       \State \textbf{return} $in + \text{table}[\gamma]$
     \Else
       \State $\gamma \gets \gamma \cdot g^{-n}$
     \EndIf
   \EndFor
     
  \end{algorithmic}
\end{algorithm}

\subsubsection{Correctness}

To prove that we always find a solution we need to show that there is always a match.

\begin{proof}
  We rewrite $x$ as follows:
  \[ x = nq + r, 0 \leq r < n \]
  We can solve for q and since $n > \sqrt{N}$ we obtain:
  \[ q = \frac{x-r}{n} < \frac{N}{n} < \frac{N}{\sqrt{N}} = \sqrt{N} < n \]
  Rewriting the discrete logarithm we get:
  \[ g^r = h\cdot g^{-nq} \]
  And since $0 \leq r < n$, $g^r$ is in the babysteps and because $q < n$, $-nq < -n^2$, so $h\cdot g^{-nq}$ is in the giant steps, therefore we always find a solution.
  
\end{proof}

\subsubsection{Time/Space Complexity}
Since we have $\mathcal{O}(1)$ lookup into the hashtable we have time complexity $\mathcal{O}(n) = \mathcal{O}(\sqrt{n})$. This consists of $2n$ modular exponentiations and $n$ hash table inserts and lookups respectively.

Regarding space, we have a hash table storing $n + 1$ elements, therefore we have $\mathcal{O}(\sqrt{n})$ space complexity.

\section{Pohlig-Hellman Algorithm}

The idea is rather simple. We have a group $G$ and want to find the discrete logarithm of a generator element $g \in G$, for some $h \in $, i.e. solve $g^x \equiv h \pmod{N}$, where $N = |G|$ is the order of $G$. We write $N = q_1^{e_1}\dots q_n^{e_n}$ where the $q_i$ are the prime factors of $N$. We now look for solutions $y_i$ for the sub-problems
\[ g_i^{y} \equiv h_i \pmod{q_i^{e_i}} \]
where $g_i = g^{N/{q_i^{e_i}}}, h_i = h^{N/q_i^{e_i}}$.
And find a solution $x$ as a solution of the resulting system of congruence relations (by the chinese remainder theorem).

\subsubsection{Solving the Sub-Problems}

The naive approach to finding the $y_i$s is of course to just apply some discrete logarithm algorithm, e.g. the babystep-giantstep algorithm described above. However, we can use a clever idea to speed up the process. To solve the discrete logarithm problem for an element of order $q^e$ where q is a prime, we express an unknown solution as follows: $x = x_0 + x_1q + x_2q^2 + \dots + x_{e-1}q^{e-1}$. Therefore we have
\begin{align*}
h^{q^{e-1}} &\equiv (g^x)^{q^{e-1}} \\
  &\equiv (g^{x_0 + x_1q + x_2q^2 + \dots + x_{e-1}q^{e-1}})^{q^{e-1}} \\
  &\equiv (g^{x_0})^{q^{e-1}}(g^{q^e})^{x_1 + x_2q + \dots + x_{e-1}q^{e-2}} \\
  &\equiv (g^{q^{e-1}})^{x_0}
\end{align*}

Thus we can use the babystep-giantstep algorithm to solve $h^{q^{e-1}} \equiv (g^{q^{e-1}})^x \pmod{p}$. We get to $x_1$ by the same principle:

\begin{align*}
  h^{q^{e-2}}
  &\equiv (g^{x_0 + x_1q + x_2q^2 + \dots + x_{e-1}q^{e-1}})^{q^{e-2}} \\
  &\equiv (g^{x_0 + x_1q})^{q^{e-2}}(g^{q^e})^{x_2 + \dots + x_{e-1}q^{e-3}} \\
  (hg^{-x_0})^{q^{e-2}} &\equiv (g^{q^{e-1}})^{x_1} \\
\end{align*}

And by following the same process we get the following definition for the i-th equation:

\[ (g^{q^{e-1}})^{x_i} \equiv (hg^{-x_0 - x_1q - \dots - x_{i-1}q^{i-1}})^{q^{e-i-1}}\].

So, in order to solve the discrete logarithm for an element of order $q^e$ we calculate all the $x_i$ using the babystep-giantstep algorithm (or any other algorithm that is able to solve discrete log for elements with prime order), and combine the results to form the solution $x = x_0 + x_1q + \dots + x_{e-1}q^{e-1}$.

\section{Index Calculus Algorithm}