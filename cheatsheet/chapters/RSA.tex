
%%% Local Variables:
%%% mode: latex
%%% TeX-master: "../cryptography"
%%% End:

\chapter{RSA}
\label{chap:rsa}

Public-Key Cryptosytem based on the infeasibility of the integer factorization problem. Encrypt messages via modular exponentiation to a public key (consisting of an exponent and a modulus) and decrypt via a private key derived from the public key and the primes used to generate it.

\section{Key Generation}
\begin{enumerate}
  \item Choose (random) large primes $p, q$
  \item Calculate $N = pq$ and $\lambda(N) = \lcm(\phi(p), \phi(q)) = \lcm(p-1, q-1)$
  \item Choose a public exponent $3 \leq e < \lambda(N)$ with $gcd(e, \lambda(N)) = 1$
  \item Calculate the private exponent $d \equiv e^{-1} \pmod{\lambda(N)}$
\end{enumerate}

\subsection{Euler's Totient Function vs Carmichael Function}

Originally RSA was formulated by using $\phi(N) = (p-1)(q-1)$ as the modulus for the inverse calculation, instead of $\lambda(N)$. Correctness for RSA is still guaranteed this way, but $gcd(e, \phi(N)) = 1$ is only a sufficient condition for a valid RSA public key, while $gcd(e, \lambda(N)) = 1$ is a necessary condition.

\begin{proof}
  Clearly $\lambda(N) | \phi(N)$, as all orders in the multiplicative group divide $\phi$ and therefore if $\lambda(N)$ were greater than $\phi(N)$ it would not be the least common multiple. It follows that any tuple $(d, e, N)$ generated using $\phi(N)$ is also valid for computations using the carmichael function, as
  \[ m^{ed} \equiv m^{1 + k\phi(N)} \equiv m^{1 + kM\lambda(N)} \equiv m^1 \pmod{N}. \]

  On the other hand some tuple $(d, e, N)$ generated using the carmichael function does not guarantee $gcd(e, \phi(N)) = 1$, as the following does not generally hold
  \[ ed \equiv 1 + k\lambda(N) \nequiv \pmod{\phi{N}} \]
  because $\lambda(N) \leq \phi(N)$ implies that $k\lambda(N)$ might not be a multiple of $\phi(N)$.
\end{proof}


\section{Encryption/Decryption}
Alice wants to send $m$ to Bob. Alice calculates
\[ c \equiv m^e \pmod{N} \]
where $(e,N)$ is Bob's public key and sends it to Bob. Bob receives $c$ and decrypts it via:
\[ c^d \equiv (m^e)^d \equiv m^{de} \equiv m \pmod{N} \]

This congruence relation works due to euler's theorem. Since we have $de \equiv 1 \pmod{\lambda(N)}$ it follows that $de = 1 + k\lambda(N)$, so $m^{de} \equiv m^{1 + k\lambda(N)} \equiv m^1m^{k\lambda(N)} \equiv m^1$.

\section{Attacks/Pitfalls}

Don't implement RSA by yourself. There are endless mistakes that immediately compromise the security of the implementation. The following will list a bunch of these and explain how to exploit them.

\subsection{Factoring N from $\lambda(N)$ or $\phi(N)$}

Assuming we somehow obtain the values for $\lambda(N)$ or $\phi(N)$ that were used when generating the RSA keys, we can obviously easily decrypt messages by calculating $d = e^{-1} \pmod{\lambda(N)}$, but can we also factor N directly? Yes! In the following we assume the standard practice of generating N as the product of two distinc primes $p,q$ was used.

\subsubsection{Knowing $\phi(N)$}

We have $\phi(N) = (p-1)(q-1) = N - p - q + 1$. So $N - (p + q) + 1 = \phi(N)$ and $p+q = N - \phi(N) + 1$. Now look at $f(x) = (x - p)(x - q) = x^2 - (p + q)x + pq$, which has its roots at $p$ and $q$. By substituting in $(p+q)$ we get an equation that only depends on $N$ and $\phi(N)$:
\[ x^2 - (N - \phi(N) + 1)x + pq \]
We obtain p and q by applying the quadratic formula:
\[ p,q = \frac{N - \phi(N) + 1}{2} \pm \sqrt{\frac{(N - \phi(N) + 1)^2}{4} - 1} \]

\subsubsection{Knowning $\lambda(N)$}

\subsection{Factoring N from d}

We can also factor $N$ if we only have the private exponent $d$. The basis for the attack are square roots of 1 modulo $N$. If we have:
\[ y^2 \equiv 1 \pmod{N} \]
\[ y^2 - 1 \equiv 0 \pmod{N} \]
\[ (y+1)(y-1) \equiv 0 \pmod{N} \]
because $p | N$ and $q | N$ and the zero product property holds, we know that:
\[ y \equiv \pm 1 \pmod{p} \;\text{and}\; y \equiv \pm 1 \pmod{q}. \]

If we find a solution to one of the following non-trivial system of equations:
\[ y \equiv 1 \pmod{p} \]
\[ y \equiv -1 \pmod{q} \]
(resp. the other non-trivial one), then we have $p | y - 1$ and $p | N$, so we can factor N via $p = gcd(y - 1, N)$.

To actually calculate the square roots we use the fact that if we have $a^b = c$ we can calculate $\sqrt{c}$ easily as $a^{\frac{b}{2}}$ if $2 | b$. Note that the following therefore only works if $2 | k$. This is certainly true if $\phi(N)$ was used during generation of the keys, because $\phi(N) = (p-1)(q-1)$ is obviously even, but I'm not sure if this is the case for the carmichael function as well. This leads us to the following algorithm:

\begin{algorithm}
  \caption{Factoring N given d}
  \begin{algorithmic}
    \Ensure $k = 2^tr, \text{r odd}$
    \State $k \gets ed - 1$
    \Loop
    \State $t \gets k$
    \State $x \gets \text{randrange}(2, N-1)$
    \While{$\text{t is even}$}
    \State $t \gets t/2$
    \State $s \gets \text{pow}(x, t, N)$
    \If{$x > 1 \; \wedge \; gcd(x - 1, N) > 1$}
    \State $p \gets gcd(x - 1, N)$
    \State \textbf{return} $(p, N/p)$
    \EndIf
    \EndWhile
    \EndLoop
  \end{algorithmic}
\end{algorithm}

There exist possible optimizations, e.g. using increasing primes instead of random choices for $x$\footnote{See this post for more info \url{https://crypto.stackexchange.com/questions/62482/algorithm-to-factorize-n-given-n-e-d}}.


\subsection{Broken RSA}
This is based on a challenge on Cryptohack. In it, a prime $p$ was used as the modulus and $e$ was not coprime to $\lambda(p) = p - 1$. If $gcd(e, p-1) = 1$ it is extremely easy to decrypt the ciphertext, as $d \equiv e^{-1} \pmod{p}$ is easy to calculate. But as this was not the case, $e$ is not invertible in $\mathbb{Z}/(p-1)\mathbb{Z}$. The trick is to divide out the common divisors of $e$ and $\lambda(n)$ and try out all posible values for $m$, which are combinations of some arbitrary solution and $e$-th roots of unity.

Basically if we define $s = p_1\dots p_k$ as the product of the shared prime factors of $e$ and $\phi(n)$, the values $a^{\frac{\phi(n)}{s}} \pmod{n}$ for arbitrary a are all $e$-th roots of unity, since
\[(a^{\frac{\phi(n)}{s}})^e \equiv a^{\frac{\phi(n)e}{s}} \equiv a^{\phi(n)\frac{e}{s}} \equiv 1 \pmod{n}\]
  because $\frac{e}{s}$ is an integer, as all factors of $s$ are also factors of $e$ by definition. It follows that all solutions of $x^e \equiv c \pmod{n}$ are of the form $x = x_0 * \zeta^i$ where $\zeta$ is a primitive $e$-th root of unity and $x_0$ is some initial solution. We obtain an initial solution easily via $x_0 \equiv c^d \pmod{n}$ with $d \equiv e^{-1} \pmod{\frac{\phi(n)}{s}}$. Clearly there are $e$ $e$-th roots of unity and the fundamental theorem of algebra tells us that $x^e - c \equiv 0$ has (at most) $e$ solutions, therefore all possible solutions are of this form.

\subsubsection{Finding the Roots of Unity}

\subsection{Small Private Exponent}

There are two algorithms here, wiener's attack and boneh's attack.

\subsubsection{Wiener's Attack}

(The following assumes the $\phi$-function was used when generating the keys)

If the private exponent is small enough, there is an efficient algorithm to recover $d$ from the public key $(e, N)$, called Wiener's attack. We know that $ed = k\phi(n) + 1$ and since $\phi(n) = pq - p - q + 1$ and $pq \gg p,q$ we can approximate $\phi(n) \approx n$. Our goal is to get to an approximation of the private key based on public information. We can get such a result by dividing $ed = k\phi(n) + 1$ by $d\phi(n)$:

\begin{align*}
  \frac{e}{\phi(n)}
  &= \frac{k\phi(n) + 1}{d\phi(n)} \\
  &= \frac{k}{d} + \frac{1}{d\phi(n)} \\
  &\approx \frac{k}{d}. \\
\end{align*}

\begin{theorem}[Wiener's Theorem]
  Let $n = pq$ with $q < p < 2q$. Let $d < \frac{1}{3}n^{\frac{1}{4}}$. Given $(e, n)$ with $ed = 1 \pmod{\phi(n)}$, Mallory can efficiently recover d.
\end{theorem}

\begin{proof}

We first note that $\abs{n - \phi(n)} < 3\sqrt{n}$, because $n - \phi(n) = p + q - 1 < p + q \leq 3q \leq 3\sqrt{n}$. 
And by using our approximation $\phi(n) \approx n$, we get $\frac{e}{n} \approx \frac{k}{d}$. We now calculate the error of this approximation:

\begin{align*}
  \left|\frac{e}{n} - \frac{k}{d}\right|
  &= \left|\frac{ed - k\phi(n) - kn + k\phi(n)}{nd}\right| \\
  &= \left|\frac{1 - k(n - \phi(n))}{nd}\right|
    \leq \left|\frac{3k\sqrt{n}}{nd}\right| = \frac{3k}{d\sqrt{n}} \\ 
\end{align*}

Now, $k\phi(n) = ed - 1 < ed$. From $e < \phi(n)$ we can deduce:
\[ k < \frac{k\phi(n)}{e} < d < \frac{1}{3}N^{\frac{1}{4}}\]
Plugging this in (using $2d < 3d < n^{\frac{1}{4}} \implies \frac{1}{2d} > \frac{1}{n^{\frac{1}{4}}}$):
\[ \left|\frac{e}{n} - \frac{k}{d}\right| \leq \frac{n^{\frac{1}{4}}}{d\sqrt{n}} = \frac{1}{dn^{\frac{1}{4}}} < \frac{1}{2d^2}. \]

Now, apparently if $\left|x - \frac{a}{b}| < \frac{1}{2b^2}$, then $\frac{a}{b}$ will be a convergent in the continued fraction expansion of $x$ and the amount of convergents between $x$ and $\frac{a}{b}$ is closely bounded by $log_2n$. Therefore, we recover $d$ in linear time (linear in the bitlength of $n$).
\end{proof}

We obtain the same approximation if we use $\lambda(n)$, there's just an extra $gcd(p-1,q-1)$ in the equations that eventually disappears.