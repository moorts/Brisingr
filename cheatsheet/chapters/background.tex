
%%% Local Variables:
%%% mode: latex
%%% TeX-master: "../cryptography"
%%% End:

\chapter{Background}

\section{Modular Arithmetic}

\subsection{Modular Exponentiation}

\begin{theorem}[Fermat's Little Theorem]
  Let $p$ be a prime number, then
  \[ a^{p} \equiv a \pmod{p}. \]
  Furthermore, if a is coprime to p, an equivalent statement is that
  \[ a^{p-1} \equiv 1 \pmod{p}. \]
\end{theorem}

\begin{theorem}[Euler's Theorem]
\label{thm:euler}
  A generalization of fermat's little theorem to arbitrary moduli. It states that for all $a$ coprime to $n$ the following holds
  \[ a^{\phi(n)} \equiv 1 \pmod{n} \]
  where $\phi$ is euler's totient function, defined as the amount of integers less than n that are coprime to n. Equivalently $\phi(n)$ is the order of the unit group of $\mathbb{Z}/n\mathbb{Z}$.
\end{theorem}

\begin{proof}
  The residue classes of the integers coprime to $n$ form a group under multiplication. The order of that group is $\phi(n)$. Let $a$ be an element of this group and $k$ be it's order. Lagrange's theorem implies that $k | \phi(n)$, so $\phi(n)$ is a multiple of $k$. We also know that $a^k \equiv 1 \pmod{n}$ per definition of the order. It follows that $a^{\phi(n)} \equiv a^{kM} \equiv (a^k)^M \equiv 1^M \equiv 1 \pmod{n}$. $m^{\lambda{N}} \equiv 1$
\end{proof}

\begin{definition}[Euler's Totient Function]
  Euler's totient function, often also called euler's phi-function, or just $\phi$ of a positive natural number $n$ is the amount of positive integers coprime to $n$ less than n, or alternatively the order of the group of units of $\mathbb{Z}/n\mathbb{Z}$.
\end{definition}

\begin{definition}[Carmichael Function]
\label{def:carmichael}
  The Carmichael function $\lambda(n)$ is defined as the smallest positive integer $m$ so that
  \[a^m \equiv 1 \pmod{n}\]
  for every $a$ coprime to $n$.
  The value of the carmichael function is also called the exponent of the multiplicative group of integers modulo n.
\end{definition}

\begin{definition}[Quadratic Residue]
  A number $q$ is called a \textbf{quadratic residue} modulo $n$ if it is congruent to a perfect square modulo $n$, meaning there exists $x$ so that:
  \[ x^2 \equiv q \pmod{n}. \]
  Otherwise, q is called a \textbf{quadratic nonresidue} modulo $n$.
\end{definition}

